\documentclass[11pt]{article}
\usepackage{amsmath}
\usepackage[utf8]{inputenc}
\usepackage[margin=1in]{geometry} 
\usepackage{hyperref}
\usepackage{longtable}
\usepackage{booktabs} 
\usepackage{ltablex} 

\title{VerdeVoyage}
\author{By: Dhairya Thakkar, Om Patel, and Enver Chowdhury}
\date{\today}

\begin{document}
\maketitle

\section*{Introduction -- Part 2}

\noindent 800 million tons. That is exactly how many tons of CO2 air travel produces every single year (IEA). In an era where global warming is a steadily increasing issue, the aviation industry stands as a significant contributor to carbon emissions. However, with millions of flights taking place annually that allow people to connect with their culture, history, and others, we simply cannot stop flying altogether. This desire for human travel that is accessible posits a critical question: How can we continue to embrace the beauty of travel without compromising the future of our planet?

\medskip

\noindent At VerdeVoyage, we are not simply asking this question, we are actively working to provide an innovative solution. Our goal is to address this challenge by providing everyone the ability to travel to their dream destination, with the lowest carbon footprint, aligning the human desire for travel with the individual responsibility to preserve our home.

\medskip

\noindent At VerdeVoyage, we take care of all our users, including those who may not know where they would like to travel, and we provide them with dream destinations through a series of questions that we ask them. We then achieve our goal through a multi-faceted approach:

\begin{itemize}
    \item Firstly, we use interactive flight graphing to map out to the user the world's airport network, and further the flights that connect these airports, allowing users to clearly see their travel options and environmental impact.
    \item Secondly, we allow the user to have balanced travel decisions where they can select flights based on factors such as price, travel time, and carbon emissions. This allows the user to have freedom on the specifics of the flight they book.
    \item Finally, we provide a tangible result of how using VerdeVoyage has benefited planet earth, along with educational tips to be eco-conscious while travelling. Providing something tangible is a crucial aspect to understanding how travelling to the same destination, but using a different flight can impact the world around you, and allows our users to be proud of their decision to become more caring.
\end{itemize}

\section*{Name and Description of All Datasets -- Part 3}

\begin{enumerate}
    \item \textbf{Flight Data with 1 Million or More Records - \href{https://www.kaggle.com/datasets/polartech/flight-data-with-1-million-or-more-records}{Kaggle Dataset}}: This is a flight dataset that contains global flight data for popular flight routes, with over one million observations and includes many columns. The only columns we use in our code include: from\_airport\_code, from\_country, dest\_airport\_code, dest\_country, aircraft\_type, airline\_name, stops, price, currency, and carbon emissions. We use all one million records in our dataset to allow our users to plan flights with the most variety at VerdeVoyage.
    
    \item \textbf {Airports Code Full} -- \href {https://data.opendatasoft.com/explore/dataset/airports-code%40public/table/}{OpenDataSoft}: This is a dataset that contains geographical information of over 9000 airports. The only columns we use from this dataset are airport\_code, latitude, and longitude. We do not use all 9000 observations. We only use the 78 rows that correspond to the 78 unique airports in our flight dataset listed above.
    
    \item \textbf {Airports Code Filtered}: This is a filtered version of the Airports Code Full dataset. It was created by us using pandas and csv to ensure that the only airports information are those that are contained in our flight dataset.
    
    \item \textbf{Country\_Traits -- ChatGPT}: This is a dataset generated by ChatGPT. At VerdeVoyage, our vacation planning process is implemented to be seamless. To do this, we ask the user a series of questions of vacation traits they look for and traverse them down a decision tree that we build and we then return the countries that match their dream. To find a dataset that answered the questions we asked, we used ChatGPT to generate a CSV file with all the destination countries in our flight data set and each column was a yes or no answer to our proposed questions.
\end{enumerate}

\section*{A Computational Overview -- Part 4}

\noindent At VerdeVoyage, our chosen domain consists of the aviation and travel industry. We focused on this industry as it has received a big boom from post-COVID, and is one of the most significant carbon emitting industries. To represent this domain, we have used varying datasets including “Flight Data with 1 Million or More Records” from Kaggle, “country\_traits” generated by personalized prompts from ChatGpt, and “Airports Code” by opendatasoft. The details of these datasets can be found in section 3. 

\medskip

\noindent The heart of VerdeVoyage’s computational strategy is a graph-based model of the world’s airport network. In this model, airports are represented as vertices (\_Vertex), each uniquely identified by an airport code and associated with a specific country. Edges between these vertices symbolize flights, representing visually to the user what countries they can travel to using VerdeVoyage. Further, we do not have a limited one edge between two vertices, we have multiple (stored as \_Vertex.neighbors) and each different edge stores information regarding ticket prices, the number of stops, and carbon emissions. Using this, we are able to display optimal routes based upon what is important to the user.

\medskip

\noindent At VerdeVoyage, we use a decision tree to ensure that any user who is indecisive about choosing their vacation destination is taken care of. Through a series of intuitive questions, we understand the user's dream destination, traverse through a decision tree, and return all the countries that match the user's preference.

\medskip

\subsection*{Computations we plan to perform:}

\begin{itemize}
    \item Our program creates a decision tree from the country\_traits.csv dataset within the build\_decision\_tree() function. This tree stores boolean values in each parent node, which corresponds to whether the user answers Y or N to a particular travel question. The leaves of this tree store country names, which are returned as suitable travel destinations for the user once the user has traversed the tree. This decision tree is very similar to the decision tree we created in exercise 2. 
    
    \item Our program creates a graph from the flight\_data.csv dataset within the create\_graph() function. It creates a new vertex for each new airport, and stores all the flights between two airports within the edge connecting them. This function is very flexible since it can produce different types of graphs based on the input parameters. Here are the four different graphs it can create:
        \begin{enumerate}
            \item A graph of all airports in the dataset.
            \item A graph of all the airports that connect to home\_airport.
            \item A graph of all the airports in dest\_countries that connect to home\_country.
            \item A graph containing only the home\_airport and the dest\_airport, and the flights connecting them.
        \end{enumerate}
    
    \item We discovered a dataset that mapped each airport code to its latitude and longitude coordinates, which we thought would be useful to our project for producing better, more accurate graph visualizations. To extract this data, we first created a new csv file that only stored the airport codes that were present in our flight\_data dataset, with their corresponding coordinates. We accomplished this by using functions in the pandas library. Then, in the get\_airport\_coordinates() function, we loaded in this data and stored it in a dictionary, mapping airport code to its coordinates. Ultimately, we stored the coordinates of each airport in its \_Vertex object, in a new attribute called coordinates.
    
    \item Our program visualizes multiple edges between two vertices. To make the paths between two vertices distinguishable, we add a unique curve to each path. We accomplished this by first creating a series of points that represent the sin graph between 0 and pi. Since the sin function is naturally curved, the points also represent a curve. We multiply these points by a number curve\_height, which will be different for each path. This value stretches or compresses the sin graph, creating a unique path. Now, we add these points to the series of points that represent a straight path between the two vertices, which makes the path curved. Notice that the endpoints of the path will still be the same since sin(0) = sin(pi) = 0.
    
    \item Our program calculates a score for each flight based on its price, number of stops, and carbon emissions, within the calculate\_flight\_scores() function. To calculate the score, we first normalize the values for price, stops, and carbon emissions to be between 0 and 1. This makes sure that the size of the value does not affect the score. Then, we ask the user how much they care about each of these factors (price, number of stops, and carbon emissions), and use those values as weights to calculate a score for each flight. We made sure that the allowed weight for carbon emissions was greater than or equal to the weights for ticket price and number of stops, to ensure that our program suggests carbon-efficient flights. We used these flight scores to suggest the most optimal flights to the user, ones that are carbon-efficient and match the user’s preferences. This is handled by the optimal\_routes() function.
\end{itemize}

\subsection*{Visualization of results:}

\noindent Our project aims to enhance the vacation planning process by integrating Plotly for user input and for visualizing flight routes. Through an interactive interface, users will input their vacation preferences, which will then be used to recommend destinations via a decision-tree analysis. We then use Plotly to model potential flight routes to these destinations, displaying them as interactive graphs with airports and flight paths represented as vertices and edges, respectively. This approach provides users with a tailored list of destinations and a visual map of travel options, making the vacation planning both intuitive and comprehensive. We aim to simplify decision-making and offer a visually engaging planning experience, thereby transforming the traditional approach to vacation planning.

\section*{Changes to your Project Plan -- Part 6}

\noindent From the proposal to the final submission, our project specifications have changed as a result of challenges faced and new enhancements discovered. Here are some of the main changes we have made to our project:

\begin{enumerate}
    \item \textbf{Switching from NetworkX to Plotly for graph visualization}: Initially, we had decided to use the NetworkX library to create graph visualizations for our project since we were more familiar with the library through our course exercises. However, the visualization provided by NetworkX was very limited. It was a static plot with very little user interactiveness. The plot also did not provide any sense of location, so you could not tell the distance between airports, and where each airport is located. Therefore, we decided to look for other graph visualizations. This is when we discovered the interactive graph visualizations that can be produced by the plotly library, so we decided to switch over to the plotly library. The visualization produced by plotly allows users to see the distance between airports, visualize what part of the world each airport is located in, and interact with the plot in many different ways, such as moving around the globe and looking at a single path. The plotly library also provided us a way to visualize multiple edges between two airports, which was very difficult with NetworkX.
    
    \item \textbf{Optimizing flight routes based on user preferences}: Initially, we had planned to suggest flight routes with the lowest carbon emissions. However, during the project, we realized the user may also care about other factors such as ticket price and number of stops taken by the flight. So to improve the flight suggestions provided by our program, we decided to incorporate ticket price and number of stops into our optimization algorithm. We decided to ask the user how much they care about each factor (ticket price, stops, and carbon emissions), and use those values as weights to calculate a score for each flight. We made sure that the allowed weight for carbon emissions was greater than or equal to the weights for ticket price and number of stops, to ensure that our program suggests carbon-efficient flights. We used these flight scores to suggest the most optimal flights to the user, ones that are carbon-efficient and match the user’s preferences.
    
    \item \textbf{Introduction of Environmental Tips}: We made the decision to incorporate a few brief educational suggestions into our program in response to the TAs' feedback, which emphasized the value of teaching users about the environment in addition to encouraging environmentally conscientious travel. These pointers are designed to offer modest actions that people can take on a daily basis to lessen the negative environmental effects of their travels.
    
    \item \textbf{Transition in UI, from Tkinter-based to Python Console}: From our initial proposal, we transferred from a Tkinter GUI to one based within the python console. The motivating reason behind this decision was for the ability to focus on enhanced core functionalities rather than front-end aesthetics. By doing so VerdeVoyage delivers a robust and effective platform that achieves its promise of facilitating sustainable travel planning.
\end{enumerate}

\section*{Discussion Section -- Part 7}

\noindent Our program helps the user plan their dream vacation by taking in their travel preferences and providing a list of suitable destinations, which was one of our project goals. Our program allows the user to get the most optimal flights based on how much they care about the price, the number of stops, and the carbon emissions. This allows the user to explore carbon-efficient flight options and make eco-conscious decisions. We also provide the user with an indication of how their choice of a green flight makes a difference by providing concrete statistics based on the carbon emissions they saved. We want to make it very clear to the user that choosing a carbon-efficient flight truly makes a big difference in the real world. At the end, we share some eco-friendly travel tips to encourage the user to make more eco-conscious decisions when travelling in the future. Hence, we believe that we have achieved our main project goal, which was to lower one’s carbon footprint when travelling.

\medskip

\noindent Reflecting back, the journey was definitely one filled with obstacles, however, our ambition to create the ultimate user experience allowed us to design the product we have today, and we would do it all over again in a heartbeat.

\medskip

\noindent \textbf{What are some next steps for further exploration?}

\noindent To extend the scope of our program, we would provide the user with a list of hostels in their chosen destination city for them to stay at during their trip, with the booking link for each hostel. Hostels are more eco-friendly than hotels and Airbnb, since they do not provide unnecessary amenities, such as pools, reducing their environmental impact. As a result, hostels are also inexpensive. We could make use of a dataset that contains hostel listings by their location to incorporate this feature into our program. Another feature we would add to our program is providing the users with a list of eco-friendly tourist attractions in their chosen destination, such as nature hikes. On top of this, we would suggest eco-friendly transportation options to the place of attraction to reduce negative environmental impact. With these features, our program would make a greater impact in reducing carbon emissions and be a more complete travel planner.

\section*{Works Cited}

\begin{enumerate}
    \item ``Airports Code.'' Opendatasoft.com, 2024, \url{https://data.opendatasoft.com/explore/dataset/airports-code\%40public/table/}. Accessed 4 Apr. 2024.
    \item BarkingData. ``Flight Data with 1 Million or More Records.'' Kaggle.com, 2022, \url{https://www.kaggle.com/datasets/polartech/flight-data-with-1-million-or-more-records}. Accessed 4 Apr. 2024.
    \item ``Home.'' Clever Carbon, \url{https://clevercarbon.io/}.
    \item IEA, Aviation. ``Aviation - IEA.'' IEA, 2023, \url{https://www.iea.org/energy-system/transport/aviation}. Accessed 5 Mar. 2024. (IEA)
    \item ``Numpy.linspace — NumPy V1.23 Manual.'' Numpy.org, \url{https://numpy.org/doc/stable/reference/generated/numpy.linspace.html}.
    \item ``Numpy.sin — NumPy V1.24 Manual.'' Numpy.org, \url{https://numpy.org/doc/stable/reference/generated/numpy.sin.html}.
    \item OpenAI. ``Country\_Traits by ChatGPT.'' Chat.openai.com, OpenAI, 30 Nov. 2022, \url{https://chat.openai.com}.
    \item ``Plotly.graph\_objects.Figure — 5.14.1 Documentation.'' Plotly.com, \url{https://plotly.com/python-api-reference/generated/plotly.graph_objects.Figure.html}.
    \item ``Plotly.graph\_objects.Scattergeo — 5.20.0 Documentation.'' Plotly.github.io, \url{https://plotly.github.io/plotly.py-docs/generated/plotly.graph_objects.Scattergeo.html}. Accessed 4 Apr. 2024.
\end{enumerate}

\end{document}
